% Generated by Sphinx.
\documentclass[letterpaper,10pt,polish]{manual}
\usepackage[utf8]{inputenc}
\usepackage[T1]{fontenc}
\usepackage{babel}
\usepackage{times}
\usepackage[Sonny]{fncychap}
\usepackage{longtable}
\usepackage{sphinx}


\title{Knut - Aplikacja do Tworzenia Testów Documentation}
\date{April 28 2010}
\release{0.1}
\author{Wiktor Idzikowski}
\newcommand{\sphinxlogo}{}
\renewcommand{\releasename}{Wydanie}
\makeindex
\makemodindex

\makeatletter
\def\PYG@reset{\let\PYG@it=\relax \let\PYG@bf=\relax%
    \let\PYG@ul=\relax \let\PYG@tc=\relax%
    \let\PYG@bc=\relax \let\PYG@ff=\relax}
\def\PYG@tok#1{\csname PYG@tok@#1\endcsname}
\def\PYG@toks#1+{\ifx\relax#1\empty\else%
    \PYG@tok{#1}\expandafter\PYG@toks\fi}
\def\PYG@do#1{\PYG@bc{\PYG@tc{\PYG@ul{%
    \PYG@it{\PYG@bf{\PYG@ff{#1}}}}}}}
\def\PYG#1#2{\PYG@reset\PYG@toks#1+\relax+\PYG@do{#2}}

\def\PYG@tok@gd{\def\PYG@tc##1{\textcolor[rgb]{0.63,0.00,0.00}{##1}}}
\def\PYG@tok@gu{\let\PYG@bf=\textbf\def\PYG@tc##1{\textcolor[rgb]{0.50,0.00,0.50}{##1}}}
\def\PYG@tok@gt{\def\PYG@tc##1{\textcolor[rgb]{0.00,0.25,0.82}{##1}}}
\def\PYG@tok@gs{\let\PYG@bf=\textbf}
\def\PYG@tok@gr{\def\PYG@tc##1{\textcolor[rgb]{1.00,0.00,0.00}{##1}}}
\def\PYG@tok@cm{\let\PYG@it=\textit\def\PYG@tc##1{\textcolor[rgb]{0.25,0.50,0.56}{##1}}}
\def\PYG@tok@vg{\def\PYG@tc##1{\textcolor[rgb]{0.73,0.38,0.84}{##1}}}
\def\PYG@tok@m{\def\PYG@tc##1{\textcolor[rgb]{0.13,0.50,0.31}{##1}}}
\def\PYG@tok@mh{\def\PYG@tc##1{\textcolor[rgb]{0.13,0.50,0.31}{##1}}}
\def\PYG@tok@cs{\def\PYG@tc##1{\textcolor[rgb]{0.25,0.50,0.56}{##1}}\def\PYG@bc##1{\colorbox[rgb]{1.00,0.94,0.94}{##1}}}
\def\PYG@tok@ge{\let\PYG@it=\textit}
\def\PYG@tok@vc{\def\PYG@tc##1{\textcolor[rgb]{0.73,0.38,0.84}{##1}}}
\def\PYG@tok@il{\def\PYG@tc##1{\textcolor[rgb]{0.13,0.50,0.31}{##1}}}
\def\PYG@tok@go{\def\PYG@tc##1{\textcolor[rgb]{0.19,0.19,0.19}{##1}}}
\def\PYG@tok@cp{\def\PYG@tc##1{\textcolor[rgb]{0.00,0.44,0.13}{##1}}}
\def\PYG@tok@gi{\def\PYG@tc##1{\textcolor[rgb]{0.00,0.63,0.00}{##1}}}
\def\PYG@tok@gh{\let\PYG@bf=\textbf\def\PYG@tc##1{\textcolor[rgb]{0.00,0.00,0.50}{##1}}}
\def\PYG@tok@ni{\let\PYG@bf=\textbf\def\PYG@tc##1{\textcolor[rgb]{0.84,0.33,0.22}{##1}}}
\def\PYG@tok@nl{\let\PYG@bf=\textbf\def\PYG@tc##1{\textcolor[rgb]{0.00,0.13,0.44}{##1}}}
\def\PYG@tok@nn{\let\PYG@bf=\textbf\def\PYG@tc##1{\textcolor[rgb]{0.05,0.52,0.71}{##1}}}
\def\PYG@tok@no{\def\PYG@tc##1{\textcolor[rgb]{0.38,0.68,0.84}{##1}}}
\def\PYG@tok@na{\def\PYG@tc##1{\textcolor[rgb]{0.25,0.44,0.63}{##1}}}
\def\PYG@tok@nb{\def\PYG@tc##1{\textcolor[rgb]{0.00,0.44,0.13}{##1}}}
\def\PYG@tok@nc{\let\PYG@bf=\textbf\def\PYG@tc##1{\textcolor[rgb]{0.05,0.52,0.71}{##1}}}
\def\PYG@tok@nd{\let\PYG@bf=\textbf\def\PYG@tc##1{\textcolor[rgb]{0.33,0.33,0.33}{##1}}}
\def\PYG@tok@ne{\def\PYG@tc##1{\textcolor[rgb]{0.00,0.44,0.13}{##1}}}
\def\PYG@tok@nf{\def\PYG@tc##1{\textcolor[rgb]{0.02,0.16,0.49}{##1}}}
\def\PYG@tok@si{\let\PYG@it=\textit\def\PYG@tc##1{\textcolor[rgb]{0.44,0.63,0.82}{##1}}}
\def\PYG@tok@s2{\def\PYG@tc##1{\textcolor[rgb]{0.25,0.44,0.63}{##1}}}
\def\PYG@tok@vi{\def\PYG@tc##1{\textcolor[rgb]{0.73,0.38,0.84}{##1}}}
\def\PYG@tok@nt{\let\PYG@bf=\textbf\def\PYG@tc##1{\textcolor[rgb]{0.02,0.16,0.45}{##1}}}
\def\PYG@tok@nv{\def\PYG@tc##1{\textcolor[rgb]{0.73,0.38,0.84}{##1}}}
\def\PYG@tok@s1{\def\PYG@tc##1{\textcolor[rgb]{0.25,0.44,0.63}{##1}}}
\def\PYG@tok@gp{\let\PYG@bf=\textbf\def\PYG@tc##1{\textcolor[rgb]{0.78,0.36,0.04}{##1}}}
\def\PYG@tok@sh{\def\PYG@tc##1{\textcolor[rgb]{0.25,0.44,0.63}{##1}}}
\def\PYG@tok@ow{\let\PYG@bf=\textbf\def\PYG@tc##1{\textcolor[rgb]{0.00,0.44,0.13}{##1}}}
\def\PYG@tok@sx{\def\PYG@tc##1{\textcolor[rgb]{0.78,0.36,0.04}{##1}}}
\def\PYG@tok@bp{\def\PYG@tc##1{\textcolor[rgb]{0.00,0.44,0.13}{##1}}}
\def\PYG@tok@c1{\let\PYG@it=\textit\def\PYG@tc##1{\textcolor[rgb]{0.25,0.50,0.56}{##1}}}
\def\PYG@tok@kc{\let\PYG@bf=\textbf\def\PYG@tc##1{\textcolor[rgb]{0.00,0.44,0.13}{##1}}}
\def\PYG@tok@c{\let\PYG@it=\textit\def\PYG@tc##1{\textcolor[rgb]{0.25,0.50,0.56}{##1}}}
\def\PYG@tok@mf{\def\PYG@tc##1{\textcolor[rgb]{0.13,0.50,0.31}{##1}}}
\def\PYG@tok@err{\def\PYG@bc##1{\fcolorbox[rgb]{1.00,0.00,0.00}{1,1,1}{##1}}}
\def\PYG@tok@kd{\let\PYG@bf=\textbf\def\PYG@tc##1{\textcolor[rgb]{0.00,0.44,0.13}{##1}}}
\def\PYG@tok@ss{\def\PYG@tc##1{\textcolor[rgb]{0.32,0.47,0.09}{##1}}}
\def\PYG@tok@sr{\def\PYG@tc##1{\textcolor[rgb]{0.14,0.33,0.53}{##1}}}
\def\PYG@tok@mo{\def\PYG@tc##1{\textcolor[rgb]{0.13,0.50,0.31}{##1}}}
\def\PYG@tok@mi{\def\PYG@tc##1{\textcolor[rgb]{0.13,0.50,0.31}{##1}}}
\def\PYG@tok@kn{\let\PYG@bf=\textbf\def\PYG@tc##1{\textcolor[rgb]{0.00,0.44,0.13}{##1}}}
\def\PYG@tok@o{\def\PYG@tc##1{\textcolor[rgb]{0.40,0.40,0.40}{##1}}}
\def\PYG@tok@kr{\let\PYG@bf=\textbf\def\PYG@tc##1{\textcolor[rgb]{0.00,0.44,0.13}{##1}}}
\def\PYG@tok@s{\def\PYG@tc##1{\textcolor[rgb]{0.25,0.44,0.63}{##1}}}
\def\PYG@tok@kp{\def\PYG@tc##1{\textcolor[rgb]{0.00,0.44,0.13}{##1}}}
\def\PYG@tok@w{\def\PYG@tc##1{\textcolor[rgb]{0.73,0.73,0.73}{##1}}}
\def\PYG@tok@kt{\def\PYG@tc##1{\textcolor[rgb]{0.56,0.13,0.00}{##1}}}
\def\PYG@tok@sc{\def\PYG@tc##1{\textcolor[rgb]{0.25,0.44,0.63}{##1}}}
\def\PYG@tok@sb{\def\PYG@tc##1{\textcolor[rgb]{0.25,0.44,0.63}{##1}}}
\def\PYG@tok@k{\let\PYG@bf=\textbf\def\PYG@tc##1{\textcolor[rgb]{0.00,0.44,0.13}{##1}}}
\def\PYG@tok@se{\let\PYG@bf=\textbf\def\PYG@tc##1{\textcolor[rgb]{0.25,0.44,0.63}{##1}}}
\def\PYG@tok@sd{\let\PYG@it=\textit\def\PYG@tc##1{\textcolor[rgb]{0.25,0.44,0.63}{##1}}}

\def\PYGZbs{\char`\\}
\def\PYGZus{\char`\_}
\def\PYGZob{\char`\{}
\def\PYGZcb{\char`\}}
\def\PYGZca{\char`\^}
% for compatibility with earlier versions
\def\PYGZat{@}
\def\PYGZlb{[}
\def\PYGZrb{]}
\makeatother

\begin{document}
\shorthandoff{"}
\maketitle
\tableofcontents
\hypertarget{--doc-index}{}


Jest to dokumentacja techniczna przeznaczona dla programistów, którzy chcą zrozumieć jak działa Knut - edytor testów. Składa się ona z opisu klas, atrybutów i metod.

Program został napisany w języku \href{http://python.org}{Python}, przy użyciu wrappera \href{http://www.pygtk.org}{PyGTK} biblioteki \href{http://gtk.org}{GTK+}. Do części okienek wykorzystywany był kreator interfejsów użytkownika \href{http://glade.gnome.org}{Glade}.

Zawartość:
\begin{itemize}
\item {} 
\emph{Knut - Główna klasa z logiką programu}

\item {} 
\emph{QuestionFrame - Klasa ramki pytania}

\item {} 
\emph{AnswerFrame - Klasa ramki w której znajdują się możliwe odpowiedzi}

\item {} 
\emph{Opis klas bazy danych}
\begin{itemize}
\item {} 
\emph{Klasa opisująca tabelę testów}

\item {} 
\emph{Klasa opisująca tabelę elementów testu}

\item {} 
\emph{Klasa opisująca tabelę pytań}

\item {} 
\emph{Klasa opisująca tabelę możliwych odpowiedzi}

\end{itemize}

\end{itemize}
\hypertarget{knut}{}

\chapter{Knut - Główna klasa z logiką programu}
\index{Knut (moduł)}
\hypertarget{module-Knut}{}
\declaremodule[Knut]{}{Knut}
\modulesynopsis{}\index{Knut (w klasie Knut)}

\hypertarget{Knut.Knut}{}\begin{classdesc}{Knut}{}
Knut - Główna klasa

Uruchamia okno główne i wczytuje konfiguracje glade

\_\_init\_\_()

Inicjalizacja okna głównego i bazy danych
\index{TEST\_CATEGORIES (Knut.Knut atrybut)}

\hypertarget{Knut.Knut.TEST\_CATEGORIES}{}\begin{memberdesc}{TEST\_CATEGORIES}
Lista dostępnych kategorii testów:
\begin{itemize}
\item {} 
Różności

\item {} 
Matematyka

\item {} 
Informatyka

\item {} 
Geografia

\item {} 
Historia

\end{itemize}
\end{memberdesc}
\index{add\_or\_update\_item() (Knut.Knut metoda)}

\hypertarget{Knut.Knut.add\_or\_update\_item}{}\begin{methoddesc}{add\_or\_update\_item}{}
Aktualizuje lub tworzy aktualnie widoczne pytanie w bazie danych
\end{methoddesc}
\index{answers\_download() (Knut.Knut metoda)}

\hypertarget{Knut.Knut.answers\_download}{}\begin{methoddesc}{answers\_download}{widget, data}
Pobieranie odpowiedzi danego ucznia
\begin{description}
\item[Argumenty:] \leavevmode\begin{itemize}
\item {} 
widget - widżet, który wywołał metodę

\item {} 
data  - dodatkowe dane zdefiniowane przy łączeniu

\end{itemize}

\end{description}
\end{methoddesc}
\index{checkdb() (Knut.Knut metoda)}

\hypertarget{Knut.Knut.checkdb}{}\begin{methoddesc}{checkdb}{}
Sprawdza czy istnieje baza danych
\begin{itemize}
\item {} 
jeśli istnieje to przygotowuje połączenie

\item {} 
jeśli bazy nie ma to tworzy tabele i łaczy z bazą

\end{itemize}
\end{methoddesc}
\index{clearMainVbox() (Knut.Knut metoda)}

\hypertarget{Knut.Knut.clearMainVbox}{}\begin{methoddesc}{clearMainVbox}{}
Usuwa widżety z głównego okna programu
\end{methoddesc}
\index{destroy\_main\_window() (Knut.Knut metoda)}

\hypertarget{Knut.Knut.destroy\_main\_window}{}\begin{methoddesc}{destroy\_main\_window}{widget, data=None}
Metoda uruchamiana przy zdarzeniu zamknięcia programu

Zapisuje ustawienia serwera do pliku settings.txt
\begin{description}
\item[Argumenty] \leavevmode\begin{itemize}
\item {} 
widget - widżet, który wywołał metodę

\item {} 
data - opcjonalnie, dodatkowe dane zdefiniowane przy łączeniu

\end{itemize}

\end{description}
\end{methoddesc}
\index{get\_current\_test() (Knut.Knut metoda)}

\hypertarget{Knut.Knut.get\_current\_test}{}\begin{methoddesc}{get\_current\_test}{offset}
Zapisuje aktualnie zaznaczony test w zmiennej self.test
\begin{description}
\item[Argumenty] \leavevmode\begin{itemize}
\item {} 
offset - offset przy stronicowaniu np. na drugiej stronie wybieramy 3 test czyli 13 w bazie danych

\end{itemize}

\end{description}
\end{methoddesc}
\index{get\_item\_type() (Knut.Knut metoda)}

\hypertarget{Knut.Knut.get\_item\_type}{}\begin{methoddesc}{get\_item\_type}{id}
Zamienia id typu odpowiedzi na zmienną tekstową
\begin{description}
\item[Argumenty] \leavevmode\begin{itemize}
\item {} 
id - liczba całkowita:

\end{itemize}

\item[Wartości zwracane] \leavevmode\begin{itemize}
\item {} 
1 -\textgreater{} `one' - pytanie jednokrotnego wyboru

\item {} 
2 -\textgreater{} `mul' - pytanie wielokrotnego wyboru

\item {} 
3 -\textgreater{} `t/f' - pytanie typu prawda/fałsz

\end{itemize}

\end{description}
\end{methoddesc}
\index{get\_selected\_results() (Knut.Knut metoda)}

\hypertarget{Knut.Knut.get\_selected\_results}{}\begin{methoddesc}{get\_selected\_results}{offset}
Zapisuje id aktualnie zaznaczonego ucznia w zmiennej self.result\_username
\begin{description}
\item[Argumenty] \leavevmode\begin{itemize}
\item {} 
offset - offset przy stronicowaniu np. na drugiej stronie wybieramy 3 wyniki czyli 13 w bazie danych

\end{itemize}

\end{description}
\end{methoddesc}
\index{load\_server\_conf() (Knut.Knut metoda)}

\hypertarget{Knut.Knut.load\_server\_conf}{}\begin{methoddesc}{load\_server\_conf}{}
Wczytuje konfiguracje serwera z pliku settings.txt do zmiennej self.server\_conf
\end{methoddesc}
\index{next\_btn\_clicked() (Knut.Knut metoda)}

\hypertarget{Knut.Knut.next\_btn\_clicked}{}\begin{methoddesc}{next\_btn\_clicked}{widget, data=None}
Zmienia aktualnie wyświetlane pytanie na następne
\begin{description}
\item[Argumenty] \leavevmode\begin{itemize}
\item {} 
widget - widżet, który wywołał metodę

\item {} 
data - opcjonalnie, dodatkowe dane zdefiniowane przy łączeniu

\end{itemize}

\end{description}
\end{methoddesc}
\index{post\_files() (Knut.Knut metoda)}

\hypertarget{Knut.Knut.post\_files}{}\begin{methoddesc}{post\_files}{tar\_file\_path, answers\_file\_path}
Wysyła archiwum z testem i plik xml z odpowiedziami na serwer
\begin{description}
\item[Argumenty] \leavevmode\begin{itemize}
\item {} 
tar\_file\_path - zmienna tekstowa, ścieżka do archiwum testu

\item {} 
answers\_file\_path - zmienna tekstowa, ścieżka do pliku z odpowiedziami

\end{itemize}

\end{description}
\end{methoddesc}
\index{prepare\_img() (Knut.Knut metoda)}

\hypertarget{Knut.Knut.prepare\_img}{}\begin{methoddesc}{prepare\_img}{img\_path, dir\_path, prefix}
Kopiuje obrazek do katalogu dla danego testu i pytania
\begin{description}
\item[Argumenty] \leavevmode\begin{itemize}
\item {} 
img\_path - zmienna tekstowa, ścieżka do obrazka

\item {} 
dir\_path - zmienna tekstowa, ścieżka do katalogu, do którego obrazek bedzie kopiowany

\item {} 
prefix - prefix nazwy pliku obrazka w celu uniknięcia nadpisania obrazków o tej samej nazwie

\end{itemize}

\end{description}
\end{methoddesc}
\index{prev\_btn\_clicked() (Knut.Knut metoda)}

\hypertarget{Knut.Knut.prev\_btn\_clicked}{}\begin{methoddesc}{prev\_btn\_clicked}{widget, data=None}
Zmienia aktualnie wyświetlane pytanie na poprzednie
\begin{description}
\item[Argumenty] \leavevmode\begin{itemize}
\item {} 
widget - widżet, który wywołał metodę

\item {} 
data - opcjonalnie, dodatkowe dane zdefiniowane przy łączeniu

\end{itemize}

\end{description}
\end{methoddesc}
\index{read\_config() (Knut.Knut metoda)}

\hypertarget{Knut.Knut.read\_config}{}\begin{methoddesc}{read\_config}{existingConfig, warning=None}
Otwiera okno dialogowe, w którym podaje się ustawienia testu
\begin{description}
\item[Argumenty] \leavevmode\begin{itemize}
\item {} 
existingConfig - obiekt testu, danymi z tego testu uzupełniane są pola okna dialogowego

\item {} 
warning - opcjonalnie, zmienna logiczna, jeśli warning=True to wyświetla ostrzeżenie o nie wypełnieniu wszystkich pól

\end{itemize}

\end{description}
\end{methoddesc}
\index{save\_error() (Knut.Knut metoda)}

\hypertarget{Knut.Knut.save\_error}{}\begin{methoddesc}{save\_error}{text}
W przypadku błędu w komunikacji z serwerem zapisuje odpowiedź serwera do pliku error.html
\begin{description}
\item[Argumenty] \leavevmode\begin{itemize}
\item {} 
text - zmienna tekstowa, treść błędu

\end{itemize}

\end{description}
\end{methoddesc}
\index{show\_about\_window() (Knut.Knut metoda)}

\hypertarget{Knut.Knut.show\_about\_window}{}\begin{methoddesc}{show\_about\_window}{widget}
Otwiera okno z podstawowymi informacjami o programie
\begin{description}
\item[Argumenty] \leavevmode\begin{itemize}
\item {} 
widget - widżet, który wywołał metodę

\end{itemize}

\end{description}
\end{methoddesc}
\index{show\_item() (Knut.Knut metoda)}

\hypertarget{Knut.Knut.show\_item}{}\begin{methoddesc}{show\_item}{item=None}
Wyświetla bierzące pytanie w oknie głównym programu
\begin{description}
\item[Argumenty] \leavevmode\begin{itemize}
\item {} 
item - obiekt bierzącego pytania

\end{itemize}

\end{description}
\end{methoddesc}
\index{show\_msg() (Knut.Knut metoda)}

\hypertarget{Knut.Knut.show\_msg}{}\begin{methoddesc}{show\_msg}{msg}
Otwiera okno dialogowe
\begin{description}
\item[Argumenty] \leavevmode\begin{itemize}
\item {} 
msg - zmienna tekstowa, wiadomość do wyświetlenia w oknie

\end{itemize}

\end{description}
\end{methoddesc}
\index{show\_server\_config\_window() (Knut.Knut metoda)}

\hypertarget{Knut.Knut.show\_server\_config\_window}{}\begin{methoddesc}{show\_server\_config\_window}{warning=None}
Otwiera okno z danymi serwera
\begin{description}
\item[Argumenty] \leavevmode\begin{itemize}
\item {} 
warning - opcjonalnie, zmienna logiczna - jeśli waning=True pokazuje ostrzeżenie, że wszystkie pola muszą być wypełnione

\end{itemize}

\end{description}
\end{methoddesc}
\index{test\_browse() (Knut.Knut metoda)}

\hypertarget{Knut.Knut.test\_browse}{}\begin{methoddesc}{test\_browse}{widget=None, data=0}
Przeglądanie testów z bazy danych
\begin{description}
\item[Argumenty] \leavevmode\begin{itemize}
\item {} 
widget - opcjonalnie, widżet, który wywołał metodę

\item {} 
data  - opcjonalnie, dodatkowe dane zdefiniowane przy łączeniu

\end{itemize}

\end{description}
\end{methoddesc}
\index{test\_browse\_from\_server() (Knut.Knut metoda)}

\hypertarget{Knut.Knut.test\_browse\_from\_server}{}\begin{methoddesc}{test\_browse\_from\_server}{widget=None, data=0, title='', type=''}
Wyświetla dane pobrane z serwera w formie tabelki
\begin{description}
\item[Argumenty] \leavevmode\begin{itemize}
\item {} 
widget - opcjonalnie, widżet, który wywołał metodę

\item {} 
data  - opcjonalnie, dodatkowe dane zdefiniowane przy łączeniu

\item {} 
title - opcjonalnie, tytuł wyświetlany nad tabelką

\item {} 
type - opcjonalnie, typ tabelki (wyniki, odpowiedzi bądź testy)

\end{itemize}

\end{description}
\end{methoddesc}
\index{test\_delete() (Knut.Knut metoda)}

\hypertarget{Knut.Knut.test\_delete}{}\begin{methoddesc}{test\_delete}{widget=None, data=None}
Otwiera okno z potwierdzeniem usunięcia testu

Po potwierdzeniu usuwa z bazy i dysku wybrany test
\begin{description}
\item[Argumenty] \leavevmode\begin{itemize}
\item {} 
widget - opcjonalnie, widżet, który wywołał metodę

\item {} 
data  - opcjonalnie, dodatkowe dane zdefiniowane przy łączeniu

\end{itemize}

\end{description}
\end{methoddesc}
\index{test\_delete\_from\_server() (Knut.Knut metoda)}

\hypertarget{Knut.Knut.test\_delete\_from\_server}{}\begin{methoddesc}{test\_delete\_from\_server}{widget, data}
Wysyła żądanie usunięcia testu do serwera

Po otrzymaniu odpowiedzi wyświetla komunikat o statusie żądania
\begin{description}
\item[Argumenty] \leavevmode\begin{itemize}
\item {} 
widget - widżet, który wywołał metodę

\item {} 
data  - dodatkowe dane zdefiniowane przy łączeniu

\end{itemize}

\end{description}
\end{methoddesc}
\index{test\_download() (Knut.Knut metoda)}

\hypertarget{Knut.Knut.test\_download}{}\begin{methoddesc}{test\_download}{widget, data}
Pobiera z serwera test i zapisuje go do bazy danych i na dysku
\begin{description}
\item[Argumenty] \leavevmode\begin{itemize}
\item {} 
widget - widżet, który wywołał metodę

\item {} 
data  - dodatkowe dane zdefiniowane przy łączeniu

\end{itemize}

\end{description}
\end{methoddesc}
\index{test\_download\_results\_list() (Knut.Knut metoda)}

\hypertarget{Knut.Knut.test\_download\_results\_list}{}\begin{methoddesc}{test\_download\_results\_list}{widget, data}
Pobranie listy wynikow dla danego ucznia
\begin{description}
\item[Argumenty] \leavevmode\begin{itemize}
\item {} 
widget - widżet, który wywołał metodę

\item {} 
data  - dodatkowe dane zdefiniowane przy łączeniu

\end{itemize}

\end{description}
\end{methoddesc}
\index{test\_edit\_settings() (Knut.Knut metoda)}

\hypertarget{Knut.Knut.test\_edit\_settings}{}\begin{methoddesc}{test\_edit\_settings}{widget=None, data=None}
Otwiera okno umożliwijące edycje ustawień testu
\begin{description}
\item[Argumenty] \leavevmode\begin{itemize}
\item {} 
widget - opcjonalnie, widżet, który wywołał metodę

\item {} 
data  - opcjonalnie, dodatkowe dane zdefiniowane przy łączeniu

\end{itemize}

\end{description}
\end{methoddesc}
\index{test\_list\_download() (Knut.Knut metoda)}

\hypertarget{Knut.Knut.test\_list\_download}{}\begin{methoddesc}{test\_list\_download}{widget=None, data=None}
Pobiera listę testów z serwera
\begin{description}
\item[Argumenty] \leavevmode\begin{itemize}
\item {} 
widget - opcjonalnie, widżet, który wywołał metodę

\item {} 
data  - opcjonalnie, dodatkowe dane zdefiniowane przy łączeniu

\end{itemize}

\end{description}
\end{methoddesc}
\index{test\_new() (Knut.Knut metoda)}

\hypertarget{Knut.Knut.test\_new}{}\begin{methoddesc}{test\_new}{widget=None, data=None}
Tworzy nowy test
\begin{description}
\item[Argumenty] \leavevmode\begin{itemize}
\item {} 
widget - opcjonalnie, widżet, który wywołał metodę

\item {} 
data  - opcjonalnie, dodatkowe dane zdefiniowane przy łączeniu

\end{itemize}

\end{description}
\end{methoddesc}
\index{test\_open() (Knut.Knut metoda)}

\hypertarget{Knut.Knut.test\_open}{}\begin{methoddesc}{test\_open}{widget=None, data=None}
Otwiera wybrany test
\begin{description}
\item[Argumenty] \leavevmode\begin{itemize}
\item {} 
widget - opcjonalnie, widżet, który wywołał metodę

\item {} 
data  - opcjonalnie, dodatkowe dane zdefiniowane przy łączeniu

\end{itemize}

\end{description}
\end{methoddesc}
\index{test\_print() (Knut.Knut metoda)}

\hypertarget{Knut.Knut.test\_print}{}\begin{methoddesc}{test\_print}{widget=None, data=None}
Drukuje wybrany test
\begin{description}
\item[Argumenty] \leavevmode\begin{itemize}
\item {} 
widget - opcjonalnie, widżet, który wywołał metodę

\item {} 
data  - opcjonalnie, dodatkowe dane zdefiniowane przy łączeniu

\end{itemize}

\end{description}
\end{methoddesc}
\index{test\_upload() (Knut.Knut metoda)}

\hypertarget{Knut.Knut.test\_upload}{}\begin{methoddesc}{test\_upload}{widget=None, data=None}
Zapisuje wybrany test do pliku xml, kompresuje test i oddzielnie odpowiedzi

Przygotowuje plik do wysłania na serwer
\begin{description}
\item[Argumenty] \leavevmode\begin{itemize}
\item {} 
widget - opcjonalnie, widżet, który wywołał metodę

\item {} 
data - opcjonalnie, dodatkowe dane zdefiniowane przy łączeniu

\end{itemize}

\end{description}
\end{methoddesc}
\index{validate\_input() (Knut.Knut metoda)}

\hypertarget{Knut.Knut.validate\_input}{}\begin{methoddesc}{validate\_input}{}
Sprawdza czy aktualnie edytowane pytanie ma wypełnione wszystkie wymagane pola
\end{methoddesc}
\end{classdesc}
\hypertarget{questionframe}{}

\chapter{QuestionFrame - Klasa ramki pytania}
\index{QuestionFrame (moduł)}
\hypertarget{module-QuestionFrame}{}
\declaremodule[QuestionFrame]{}{QuestionFrame}
\modulesynopsis{}\index{QuestionFrame (w klasie QuestionFrame)}

\hypertarget{QuestionFrame.QuestionFrame}{}\begin{classdesc}{QuestionFrame}{item}
Klasa ramki pytania

\_\_init\_\_(item)

Inicjalizuje okno pytania, uzupełnia pytanie danymi ze zmiennej item
\begin{description}
\item[Argumenty] \leavevmode\begin{itemize}
\item {} 
item - obiekt bierzącego pytania, jeśli brak to wyświetla puste pola do edycji

\end{itemize}

\end{description}
\index{add\_img() (QuestionFrame.QuestionFrame metoda)}

\hypertarget{QuestionFrame.QuestionFrame.add\_img}{}\begin{methoddesc}{add\_img}{img\_filename}
Zmienia lub wyświetla nowy obrazek
\begin{description}
\item[Argumenty] \leavevmode\begin{itemize}
\item {} 
img\_filename - ścieżka do obrazka

\end{itemize}

\end{description}
\end{methoddesc}
\index{img\_button\_clicked() (QuestionFrame.QuestionFrame metoda)}

\hypertarget{QuestionFrame.QuestionFrame.img\_button\_clicked}{}\begin{methoddesc}{img\_button\_clicked}{widget=None, data=None}
Otwiera okno wyboru pliku i wyświetla wybrany obrazek w sekcji pytania
\begin{description}
\item[Argumenty] \leavevmode\begin{itemize}
\item {} 
widget - opcjonalnie, widżet, który wywołał metodę

\item {} 
data  - opcjonalnie, dodatkowe dane zdefiniowane przy łączeniu

\end{itemize}

\end{description}
\end{methoddesc}
\index{remove\_img() (QuestionFrame.QuestionFrame metoda)}

\hypertarget{QuestionFrame.QuestionFrame.remove\_img}{}\begin{methoddesc}{remove\_img}{widget=None}
Usuwa obrazek z pytania
\begin{description}
\item[Argumenty] \leavevmode\begin{itemize}
\item {} 
widget - opcjonalnie, widżet, który wywołał metodę

\end{itemize}

\end{description}
\end{methoddesc}
\end{classdesc}
\hypertarget{answerframe}{}

\chapter{AnswerFrame - Klasa ramki w której znajdują się możliwe odpowiedzi}
\index{AnswerFrame (moduł)}
\hypertarget{module-AnswerFrame}{}
\declaremodule[AnswerFrame]{}{AnswerFrame}
\modulesynopsis{}\index{AnswerFrame (w klasie AnswerFrame)}

\hypertarget{AnswerFrame.AnswerFrame}{}\begin{classdesc}{AnswerFrame}{item=None}
Klasa ramki w której znajdują się możliwe odpowiedzi

\_\_init\_\_(item=None)

Inicjalizuje ramke odpowiedzi, uzupełnia pola danymi ze zmiennej item
\begin{description}
\item[Argumenty] \leavevmode\begin{itemize}
\item {} 
item - opcjonalnie, obiekt bierzącego pytania, jeśli brak to wyświetla puste pola do edycji

\end{itemize}

\end{description}
\index{add\_img() (AnswerFrame.AnswerFrame metoda)}

\hypertarget{AnswerFrame.AnswerFrame.add\_img}{}\begin{methoddesc}{add\_img}{img\_filename, index}
Dodaje lub zamienia obrazek przy danej odpowiedzi
\begin{description}
\item[Argumenty] \leavevmode\begin{itemize}
\item {} 
img\_filename - ścieżka do pliku obrazka

\item {} 
index - indeks odpowiedzi, w której zmienia się obrazek

\end{itemize}

\end{description}
\end{methoddesc}
\index{answer\_type\_combo\_changed() (AnswerFrame.AnswerFrame metoda)}

\hypertarget{AnswerFrame.AnswerFrame.answer\_type\_combo\_changed}{}\begin{methoddesc}{answer\_type\_combo\_changed}{widget=None, data=None}
Wyświetla pola do edycji dla danego typu odpowiedzi
\begin{description}
\item[Argumenty] \leavevmode\begin{itemize}
\item {} 
widget - opcjonalnie, widżet, który wywołał metodę

\item {} 
data - opcjonalnie, dodatkowe dane zdefiniowane przy łączeniu

\end{itemize}

\end{description}
\end{methoddesc}
\index{get\_type\_index() (AnswerFrame.AnswerFrame metoda)}

\hypertarget{AnswerFrame.AnswerFrame.get\_type\_index}{}\begin{methoddesc}{get\_type\_index}{type}
Zamienia tekst typu odpowiedzi na reprezentację liczbową
\begin{description}
\item[Argumenty] \leavevmode\begin{itemize}
\item {} 
type - zmienna tekstowa:

\end{itemize}

\item[Wartości zwracane] \leavevmode\begin{itemize}
\item {} 
`one' -\textgreater{} 1 - pytanie jednokrotnego wyboru

\item {} 
`mul' -\textgreater{} 2 - pytanie wielokrotnego wyboru

\item {} 
`t/f' -\textgreater{} 3 - pytanie typu prawda/fałsz

\end{itemize}

\end{description}
\end{methoddesc}
\index{img\_button\_clicked() (AnswerFrame.AnswerFrame metoda)}

\hypertarget{AnswerFrame.AnswerFrame.img\_button\_clicked}{}\begin{methoddesc}{img\_button\_clicked}{widget=None, data=None}
Otwiera okno wyboru obrazka do danej odpowiedzi
\begin{description}
\item[Argumenty] \leavevmode\begin{itemize}
\item {} 
widget - opcjonalnie, widżet, który wywołał metodę

\item {} 
data - opcjonalnie, dodatkowe dane zdefiniowane przy łączeniu

\end{itemize}

\end{description}
\end{methoddesc}
\index{remove\_img() (AnswerFrame.AnswerFrame metoda)}

\hypertarget{AnswerFrame.AnswerFrame.remove\_img}{}\begin{methoddesc}{remove\_img}{widget, index}
Usuwa wybrany obrazek
\begin{description}
\item[Argumenty] \leavevmode\begin{itemize}
\item {} 
widget - widżet, który wywołał metodę

\item {} 
index - indeks odpowiedzi, w której usuwa obrazek

\end{itemize}

\end{description}
\end{methoddesc}
\end{classdesc}
\hypertarget{dbmodel}{}

\chapter{Opis klas bazy danych}
\index{dbmodel (moduł)}
\hypertarget{module-dbmodel}{}
\declaremodule[dbmodel]{}{dbmodel}
\modulesynopsis{}

\hypertarget{dbmodel-test}{}\section{Klasa opisująca tabelę testów}
\index{Test (w klasie dbmodel)}

\hypertarget{dbmodel.Test}{}\begin{classdesc}{Test}{**kwargs}
Klasa opisująca tabelę testów

Opis pól:
\index{author (dbmodel.Test atrybut)}

\hypertarget{dbmodel.Test.author}{}\begin{memberdesc}{author}
Imię i nazwisko autora
\end{memberdesc}
\index{category (dbmodel.Test atrybut)}

\hypertarget{dbmodel.Test.category}{}\begin{memberdesc}{category}
Kategoria w jakiej test się znajduje
\end{memberdesc}
\index{instructions (dbmodel.Test atrybut)}

\hypertarget{dbmodel.Test.instructions}{}\begin{memberdesc}{instructions}
Instrukcje przydatne przy rozwiązywaniu testu
\end{memberdesc}
\index{item (dbmodel.Test atrybut)}

\hypertarget{dbmodel.Test.item}{}\begin{memberdesc}{item}
Relacje jeden do wielu z tabelą elementów testu
\end{memberdesc}
\index{password (dbmodel.Test atrybut)}

\hypertarget{dbmodel.Test.password}{}\begin{memberdesc}{password}
Hasło
\end{memberdesc}
\index{time (dbmodel.Test atrybut)}

\hypertarget{dbmodel.Test.time}{}\begin{memberdesc}{time}
Czas na ukończenie testu
\end{memberdesc}
\index{title (dbmodel.Test atrybut)}

\hypertarget{dbmodel.Test.title}{}\begin{memberdesc}{title}
Tytuł testu
\end{memberdesc}
\index{version (dbmodel.Test atrybut)}

\hypertarget{dbmodel.Test.version}{}\begin{memberdesc}{version}
Wersja testu
\end{memberdesc}
\end{classdesc}
\hypertarget{dbmodel-item}{}

\section{Klasa opisująca tabelę elementów testu}
\index{Item (w klasie dbmodel)}

\hypertarget{dbmodel.Item}{}\begin{classdesc}{Item}{**kwargs}
Klasa opisująca tabelę elementów testu
\index{option (dbmodel.Item atrybut)}

\hypertarget{dbmodel.Item.option}{}\begin{memberdesc}{option}
Relacja jeden do jednego z tabelą możliwych odpowiedzi
\end{memberdesc}
\index{order (dbmodel.Item atrybut)}

\hypertarget{dbmodel.Item.order}{}\begin{memberdesc}{order}
Kolejność elementu
\end{memberdesc}
\index{question (dbmodel.Item atrybut)}

\hypertarget{dbmodel.Item.question}{}\begin{memberdesc}{question}
Relacja jeden do jednego z tabelą pytań
\end{memberdesc}
\index{test (dbmodel.Item atrybut)}

\hypertarget{dbmodel.Item.test}{}\begin{memberdesc}{test}
Relacja z wiele do jednego z testem
\end{memberdesc}
\index{type (dbmodel.Item atrybut)}

\hypertarget{dbmodel.Item.type}{}\begin{memberdesc}{type}
Typ elementu
\end{memberdesc}
\end{classdesc}
\hypertarget{dbmodel-question}{}

\section{Klasa opisująca tabelę pytań}
\index{Question (w klasie dbmodel)}

\hypertarget{dbmodel.Question}{}\begin{classdesc}{Question}{**kwargs}
Klasa opisująca tabelę pytań
\index{img (dbmodel.Question atrybut)}

\hypertarget{dbmodel.Question.img}{}\begin{memberdesc}{img}
Nazwa obrazka
\end{memberdesc}
\index{item (dbmodel.Question atrybut)}

\hypertarget{dbmodel.Question.item}{}\begin{memberdesc}{item}
Relacja wiele do jednego z tabelą elementów testu
\end{memberdesc}
\index{text (dbmodel.Question atrybut)}

\hypertarget{dbmodel.Question.text}{}\begin{memberdesc}{text}
Tekst pytania
\end{memberdesc}
\end{classdesc}
\hypertarget{dbmodel-option}{}

\section{Klasa opisująca tabelę możliwych odpowiedzi}
\index{Option (w klasie dbmodel)}

\hypertarget{dbmodel.Option}{}\begin{classdesc}{Option}{**kwargs}
Klasa opisująca tabelę możliwych odpowiedzi
\index{correct (dbmodel.Option atrybut)}

\hypertarget{dbmodel.Option.correct}{}\begin{memberdesc}{correct}
Poprawność odpowiedzi
\end{memberdesc}
\index{img (dbmodel.Option atrybut)}

\hypertarget{dbmodel.Option.img}{}\begin{memberdesc}{img}
Nazwa obrazka
\end{memberdesc}
\index{item (dbmodel.Option atrybut)}

\hypertarget{dbmodel.Option.item}{}\begin{memberdesc}{item}
Relacja wiele do jednego z tabelą elementów testu
\end{memberdesc}
\index{text (dbmodel.Option atrybut)}

\hypertarget{dbmodel.Option.text}{}\begin{memberdesc}{text}
Tekst odpowiedzi
\end{memberdesc}
\end{classdesc}


\chapter{Indeksy i tabele}
\begin{itemize}
\item {} 
\emph{Index}

\item {} 
\emph{Module Index}

\item {} 
\emph{Search Page}

\end{itemize}


\renewcommand{\indexname}{Indeks modułów}
\printmodindex
\renewcommand{\indexname}{Indeks}
\printindex
\end{document}
